\frame[plain]{\titlepage}
\frame{\frametitle{Outline}\tableofcontents}

\section{Text}

\begin{frame}
    \frametitle{Text test}
    
    Night gathers, and now my watch begins.
    It shall not end until my death.
    I shall take no wife, hold no lands, father no children.
    I shall wear no crowns and win no glory.
    I shall live and die at my post.

    \vspace{0.4cm}

    \pause

    Night gathers, and now my watch begins.
    It shall not end until my death.
    I shall take no wife, hold no lands, father no children.
    I shall wear no crowns and win no glory.
    I shall live and die at my post.
    
\end{frame}

\section{Beamer Basic}
\subsection{Beamer Blocks}

\begin{frame}
    \frametitle{Basic Blocks}

    \begin{block}{Standard Block}
        This is a standard block.
    \end{block}
    \begin{alertblock}{Alert Message}
        This block presents alert message.
    \end{alertblock}
    \begin{exampleblock}{An example of typesetting tool}
        Example: MS Word, \LaTeX{}
    \end{exampleblock}

\end{frame}

\begin{frame}
    \frametitle{Math Blocks}

    \begin{definition} 
        This is a definition.
    \end{definition}
    
    \begin{theorem} 
        This is a theorem. 
    \end{theorem}
    
    \begin{lemma} 
        This is a proof idea.
    \end{lemma}

\end{frame}

\begin{frame}
    \begin{proof} 
        This is a proof. 
    \end{proof}
    
    \begin{corollary}
        This is a corollary
    \end{corollary}
    
    \begin{example}
        This is an example 
    \end{example}

\end{frame}

\subsection{Other Environments}

\begin{frame}{Algorithm}
    \scriptsize
    \begin{algorithm}[H]
        \KwData{this text}
        \KwResult{how to write algorithm with \LaTeX2e }
        initialization\;
        \While{not at end of this document}{
            read current\;
            \eIf{understand}{
            go to next section\;
            current section becomes this one\;
            }{
            go back to the beginning of current section\;
            }
        }
        \caption{How to write algorithms
        (copied from \href{https://en.wikibooks.org/wiki/LaTeX/Algorithms}{here})}
        \end{algorithm}
\end{frame}

\begin{frame}[fragile]
    \frametitle{An Algorithm For Finding Primes Numbers.}
    \scriptsize
    \begin{verbatim}
        int main (void)
        {
            std::vector<bool> is_prime (100, true);
            for (int i = 2; i < 100; i++)
            if (is_prime[i])
            {
                std::cout << i << " ";
                for (int j = i; j < 100; is_prime [j] = false, j+=i);
            }
            return 0;
        }
    \end{verbatim}

    \vspace{-0.7cm}

    \begin{uncoverenv}
    Note the use of \verb|\alert|.
    \end{uncoverenv}
\end{frame}

\section{Beamer More}
\subsection{Split Screen}

\begin{frame}{Minipage}


\begin{minipage}{.27\textwidth}
    \begin{block}{}
        \centering itemize
    \end{block}
    \begin{itemize}
        \item test 1
        \item \textit{test 2}
        \item \textbf{test 3}
        \item \textbf{\textit{test 4}}
    \end{itemize}
\end{minipage}
\begin{minipage}{.27\textwidth}
    \begin{block}{}
        \centering enumerate
    \end{block}
    \begin{enumerate}
        \item test 1
        \item \textit{test 2}
        \item \textbf{test 3}
        \item \textbf{\textit{test 4}}
    \end{enumerate}
\end{minipage}
\begin{minipage}{.35\textwidth}
    \begin{block}{}
        \centering description
    \end{block}
    \begin{description}
        \item[normal] test 1
        \item[italic] \textit{test 2}
        \item[bold] \textbf{test 3}
        \item[italic and bold] \textbf{\textit{test 4}}
    \end{description}
\end{minipage}
\end{frame}

\begin{frame}{Columns}
    \begin{columns}
        \column{0.5\textwidth}
        This is a text in first column.
        $$E=mc^2$$
        \begin{itemize}
        \item First item
        \item Second item
        \end{itemize}
        
        \column{0.5\textwidth}
        \begin{block}{first block}
            columns achieves splitting the screen
        \end{block}
        \begin{block}{second block}
            stack block in columns
        \end{block}
        
    \end{columns}
\end{frame}

\subsection{Table}

\begin{frame}{Create Tables}
    \begin{center}
        \begin{table}[!t]  
            % \caption{Three line}
            % \label{table_time}
            \begin{tabular}{ccc}  
                \toprule   
                first&second&third\\ 
                \midrule       
                1 & 2 & 3 \\ 
                4 & 5 & 6 \\ 
                7 & 8 & 9 \\
                \bottomrule  
            \end{tabular}
        \end{table}
    \end{center}
\end{frame}

\subsection{Math}

\begin{frame}{Equation1}
    A matrix in text must be set  smaller:
    $\bigl(\begin{smallmatrix}
    a&b \\ c&d
    \end{smallmatrix} \bigr)$
    to not increase leading in a portion of text.

    \[ f(n) =
    \begin{cases}
        n/2       & \quad \text{if } n \text{ is even}\\
        -(n+1)/2  & \quad \text{if } n \text{ is odd}
    \end{cases}
    \]

    $$50 apples \times 100 apples = lots of apples^2$$
\end{frame}

\begin{frame}{Equation2}
    $$\sum_{\substack{0<i<m \\ 0<j<n }} 
      P(i,j)=\int\limits_a^b\prod P(i,j)$$

    $$P\left(A=2\middle|\frac{A^2}{B}>4\right)$$

    $$( a ), [ b ], \{ c \}, | d |, \| e \|,
    \langle f \rangle, \lfloor g \rfloor,
    \lceil h \rceil, \ulcorner i \urcorner$$
\end{frame}

\begin{frame}{Equation3}
    $$Q(\alpha)=\alpha_i\alpha_jy_iy_j(x_i\cdot x_j)$$

    $$Q(\alpha)=\alpha^i\alpha^jy^{(i)}y^{(j)}(x^i\cdot x^j)$$
    
    $$\Gamma=\beta+\alpha+\gamma+\rho$$
\end{frame}



\section*{Q\&A}
\begin{frame}
    \centering
    \vspace{2.2cm}

    \structure{\fontsize{30pt}{35pt}\selectfont\textbf{Q{\Huge\&}A}}

    \vspace{0.4cm}{\Large\itshape \faSlideshare~Looking Forward to Your Comments!}
\end{frame}