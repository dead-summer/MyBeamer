\frame[plain]{\titlepage}
\frame{\frametitle{Outline}\tableofcontents}

\section{文字环境}

\begin{frame}
    \frametitle{文字测试}
    
    长夜将至,我从今开始守望,至死方休。
    我将不娶妻、不封地、不生子。
    我将不戴宝冠,不争荣宠。
    我将尽忠职守,生死于斯。

    \vspace{0.4cm}

    \pause

    长夜将至,我从今开始守望,至死方休。
    我将不娶妻、不封地、不生子。
    我将不戴宝冠,不争荣宠。
    我将尽忠职守,生死于斯。
    
\end{frame}

\section{块和其他环境}
\subsection{块}

\begin{frame}
    \frametitle{基础块环境}

    \begin{block}{标准块}
        这是一个标准块。
    \end{block}
    \begin{alertblock}{Alert 块}
        这是一个 Alert 块
    \end{alertblock}
    \begin{exampleblock}{排版工具}
        例如: MS Word, \LaTeX{}
    \end{exampleblock}

\end{frame}

\begin{frame}
    \frametitle{数学块}

    \begin{definition} 
        定义环境。
    \end{definition}
    
    \begin{theorem} 
        定理环境。
    \end{theorem}
    
    \begin{lemma} 
        引理环境。
    \end{lemma}

\end{frame}

\begin{frame}
    \begin{proof} 
        证明环境。
    \end{proof}
    
    \begin{corollary}
        推论环境。
    \end{corollary}
    
    \begin{example}
        example 环境。
    \end{example}

\end{frame}

\subsection{其他环境}

\begin{frame}{算法环境}
    \scriptsize
    \begin{algorithm}[H]
        \KwData{本文}
        \KwResult{如何用 \LaTeX2e 书写算法伪代码}
        initialization\;
        \While{本文尚未结束}{
            观看本节\;
            \eIf{理解本节}{
            查看下一节\;
            当前章节更新为新的一节\;
            }{
            回到本节开始\;
            }
        }
        \caption{如何书写算法伪代码
        (参考 \href{https://en.wikibooks.org/wiki/LaTeX/Algorithms}{此文})}
        \end{algorithm}
\end{frame}

\begin{frame}[fragile]
    \frametitle{查找素数}
    \scriptsize
    \begin{verbatim}
        int main (void)
        {
            std::vector<bool> is_prime (100, true);
            for (int i = 2; i < 100; i++)
            if (is_prime[i])
            {
                std::cout << i << " ";
                for (int j = i; j < 100; is_prime [j] = false, j+=i);
            }
            return 0;
        }
    \end{verbatim}

    \vspace{-0.7cm}

    \begin{uncoverenv}
    注意使用 \verb|\alert|.
    \end{uncoverenv}
\end{frame}

\section{更多}
\subsection{多栏并排}

\begin{frame}{Minipage}


\begin{minipage}{.27\textwidth}
    \begin{block}{}
        \centering itemize
    \end{block}
    \begin{itemize}
        \item 测试 1
        \item \textit{测试 2}
        \item \textbf{测试 3}
        \item \textbf{\textit{测试 4}}
    \end{itemize}
\end{minipage}
\begin{minipage}{.27\textwidth}
    \begin{block}{}
        \centering enumerate
    \end{block}
    \begin{enumerate}
        \item 测试 1
        \item \textit{测试 2}
        \item \textbf{测试 3}
        \item \textbf{\textit{测试 4}}
    \end{enumerate}
\end{minipage}
\begin{minipage}{.35\textwidth}
    \begin{block}{}
        \centering description
    \end{block}
    \begin{description}
        \item[普通] 测试 1
        \item[斜体] \textit{测试 2}
        \item[粗体] \textbf{测试 3}
        \item[斜粗体] \textbf{\textit{测试 4}}
    \end{description}
\end{minipage}
\end{frame}

\begin{frame}{Columns}
    \begin{columns}
        \column{0.5\textwidth}
        左列
        $$E=mc^2$$
        \begin{itemize}
        \item 测试 1
        \item 测试 2
        \end{itemize}
        
        \column{0.5\textwidth}
        \begin{block}{测试 1}
            文字测试。
        \end{block}
        \begin{block}{测试 2}
            文字测试。
        \end{block}
        
    \end{columns}
\end{frame}

\subsection{表格}

\begin{frame}{创建表格}
    \begin{center}
        \begin{table}[!t]  
            \begin{tabular}{ccc}  
                \toprule   
                第一&第二&第三\\ 
                \midrule       
                1 & 2 & 3 \\ 
                4 & 5 & 6 \\ 
                7 & 8 & 9 \\
                \bottomrule  
            \end{tabular}
        \end{table}
    \end{center}
\end{frame}

\subsection{数学公式}

\begin{frame}{公式 1}
    行内公式:
    $\bigl(\begin{smallmatrix}
    a&b \\ c&d
    \end{smallmatrix} \bigr)$,
    行间公式:

    \[ f(n) =
    \begin{cases}
        n/2       & \quad \text{if } n \text{ is even}\\
        -(n+1)/2  & \quad \text{if } n \text{ is odd}
    \end{cases}
    \]

    $$50 apples \times 100 apples = lots of apples^2$$
\end{frame}

\begin{frame}{公式 2}
    $$\sum_{\substack{0<i<m \\ 0<j<n }} 
      P(i,j)=\int\limits_a^b\prod P(i,j)$$

    $$P\left(A=2\middle|\frac{A^2}{B}>4\right)$$

    $$( a ), [ b ], \{ c \}, | d |, \| e \|,
    \langle f \rangle, \lfloor g \rfloor,
    \lceil h \rceil, \ulcorner i \urcorner$$
\end{frame}

\begin{frame}{公式 3}
    $$Q(\alpha)=\alpha_i\alpha_jy_iy_j(x_i\cdot x_j)$$

    $$Q(\alpha)=\alpha^i\alpha^jy^{(i)}y^{(j)}(x^i\cdot x^j)$$
    
    $$\Gamma=\beta+\alpha+\gamma+\rho$$
\end{frame}



\section*{Q\&A}
\begin{frame}
    \centering
    \vspace{2.2cm}

    \structure{\fontsize{30pt}{35pt}\selectfont\textbf{Q{\Huge\&}A}}

    \vspace{0.4cm}{\Large\itshape \faSlideshare~希望老师批评指正!}
\end{frame}